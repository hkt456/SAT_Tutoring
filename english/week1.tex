
\documentclass[aspectratio=169,11pt]{beamer}
\usetheme{Madrid}
\usefonttheme{professionalfonts}
\usepackage[utf8]{inputenc}
\usepackage[T1]{fontenc}
\usepackage{lmodern}
\usepackage{multicol}
\usepackage[normalem]{ulem}


\title{SAT Writing -- Week 1}
\subtitle{Introduction to SAT Writing \& Core Grammar}
\author{Huynh Khac Tam}
\date{\today}

\begin{document}

\begin{frame}
  \titlepage
\end{frame}

\begin{frame}{How to use these slides}
  \begin{itemize}
    \item \textbf{Purpose}: These slides will help you summarize the topics that are covered in class. It will not be too detailed, but should be used as a reference for any further self-revision.
    \item \textbf{Flow}: \textbf{Concept} $\rightarrow$ \textbf{Examples} $\rightarrow$ \textbf{Mini Drills}.
    \item \textbf{Notes}: Learning from these slides alone will not be sufficient, as intensive practice and daily usage is needed for a better command of English.
  \end{itemize}
\end{frame}

\begin{frame}{SAT Writing (within Reading \& Writing Section)}
\small
\textbf{Time:} 64 minutes total (for Reading + Writing) \\
\textbf{Questions:} 54 total; Writing-focused items $\sim$27--30

\vspace{0.6em}
\textbf{Writing Domains Tested:}
\begin{itemize}
  \item \textbf{Standard English Conventions} (Grammar \& Mechanics) $\sim$20 questions
    \begin{itemize}
      \item Sentence structure, verb tense, agreement
      \item Pronouns, modifiers, punctuation (commas, colons, dashes)
    \end{itemize}
    
  \item \textbf{Expression of Ideas} (Style \& Rhetoric) $\sim$7--10 questions
    \begin{itemize}
      \item Word choice, concision, clarity
      \item Transitions, logical flow, organization
    \end{itemize}
    
\end{itemize}

\vspace{0.6em}
\textbf{Format:}
\begin{itemize}
  \item Short passages (25--150 words), 1 question per passage.
  \item Writing questions target grammar, usage, and effective style.
\end{itemize}

\vspace{0.6em}
\textbf{Scoring:} Writing contributes to the Reading \& Writing score (200--800).
\end{frame}

\begin{frame}{SAT Writing Topics (Grouped)}
\begin{multicols}{2}
\textbf{Grammar \& Conventions}
\begin{itemize}
  \item Subject–Verb Agreement
  \item Pronoun Reference
  \item Verb Tense Errors
  \item Sentence Fragments
  \item Run-ons
  \item Modifiers (dangling/misplaced)
  \item Parallelism
  \item Apostrophes
  \item Commas, Dashes, and Colons
\end{itemize}

\columnbreak

\textbf{Style \& Rhetoric}
\begin{itemize}
  \item Relevance and Purpose
  \item Word Choice
  \item Redundancy
  \item Awkward Phrasing
  \item Placement (order/clarity)
  \item Combining Sentences
  \item Transitions (sentence/paragraph)
  \item Idioms
  \item Who vs.\ Whom
  \item Shift in Point of View
  \item Data Interpretation
\end{itemize}
\end{multicols}
\end{frame}

% ===== Frame 1 =====
\begin{frame}
  \centering
  \Huge \textbf{Relative Clauses} \\[1.5em]
  \Large Details or Distraction
\end{frame}


% ===== Frame 2 =====
\begin{frame}{Relative Clauses: Definition \& Examples}
\small
\textbf{Definition:} Clauses beginning with a relative pronoun (\emph{who, whom, whose, which, that}) that add detail but are not essential to the main sentence.  

\vspace{0.5em}
\textbf{Example base sentence:} \\
The tiger ate my aunt earlier today.

\vspace{0.5em}
\textbf{Add relative clause:} \\
The tiger \underline{that was hungry} ate my aunt earlier today.  

\vspace{0.5em}
\textbf{With more phrases:} \\
\underline{After escaping}, the tiger that was hungry ate my aunt, \underline{who was nice and juicy}, earlier today.
\end{frame}

% ===== Frame 3 =====
\begin{frame}{Why Are Relative Clauses Important on the SAT?}
\begin{itemize}
  \item They add detail, but are never essential to the core sentence.
  \item On the SAT, long phrases disguise the main idea — like a boxer distracting before the punch.
  \item Key strategy: strip away relative clauses and comma phrases to find the complete sentence.
\end{itemize}

\vspace{0.5em}
\textbf{Essence of the example:}  
\begin{quote}
The tiger ate my aunt earlier today.
\end{quote}
\end{frame}

% ===== Frame 4 =====
\begin{frame}{Quick Exercise: Strip It Down}
\small
\textbf{Original:}  
\underline{After escaping}, the tiger \underline{that was hungry} ate my aunt, \underline{who was nice and juicy}, earlier today.

\vspace{0.8em}
\textbf{Task:} Cross out the nonessential parts (relative clauses, comma phrases) to reveal the core sentence.

\vspace{0.8em}
\textbf{Answer:}  
The tiger ate my aunt earlier today.
\end{frame}

\begin{frame}{Relative Clauses: Practice Sentences}
\small
\begin{enumerate}
  \setcounter{enumi}{0} % starts numbering at 2
  \item After running the Boston marathon, Jack Kunis drank all the water that was left in his bottle and fell to his knees.
  \item The lost ship and its treasure that had fallen to the bottom of the ocean were never found again.
  \item Frank, in addition to his cousins, suffers from a condition known as hyperthymestic syndrome, which prevents one from ever forgetting anything.
  \item Starting at the age of 10, Mrs.\ Smith kept a daily diary, which allowed her to recall the happy memories in life.
  \item For years the chairman remained anonymous, referred to only by initials even within his inner circles.
  \item Students whose grades are low will have to report to me, the principal of the school.
\end{enumerate}
\end{frame}


\begin{frame}{Relative Clauses: Answer Key}
\small
\begin{enumerate}
  \setcounter{enumi}{1}
  \item \sout{After running the Boston marathon}, Jack Kunis drank all the water \sout{that was left in his bottle} and fell to his knees.
  \item The lost ship and its treasure \sout{that had fallen to the bottom of the ocean} were never found again.
  \item Frank, \sout{in addition to his cousins}, suffers from a condition known as hyperthymestic syndrome, \sout{which prevents one from ever forgetting anything}.
  \item \sout{Starting at the age of 10}, Mrs.\ Smith kept a daily diary, \sout{which allowed her to recall the happy memories in life}.
  \item For years the chairman remained anonymous, \sout{referred to only by initials even within his inner circles}.
  \item Students \sout{whose grades are low} will have to report to me, \sout{the principal of the school}.
\end{enumerate}
\end{frame}

% ===== Frame 1 =====
\begin{frame}
  \centering
  \Huge \textbf{Prepositional Phrases} \\[1.5em]
  \Large What We’ll Cover Today
\end{frame}

% ===== Frame 2 =====
\begin{frame}{Definition \& Examples}
\small
\textbf{Definition:} A prepositional phrase = \textbf{preposition + noun + any attached describing phrase (this is optional)}.  
\begin{itemize}
  \item Common prepositions: \emph{about, above, across, after, against, along, among, around, at, before, between, by, for, from, in, into, of, on, over, past, through, to, under, with, without} (and many more).
  \item Prepositions almost always have a noun following them.
\end{itemize}

\textbf{Examples:}
\begin{enumerate}
  \item \underline{Throughout the living room} was the scent of \underline{fatty crabs that had expired weeks ago}.
  \item I put my sister \underline{on the diet} after it worked so well \underline{for me}.
\end{enumerate}
\end{frame}

% ===== Frame 3 =====
\begin{frame}{Why Are Prepositional Phrases Important?}
\begin{itemize}
  \item They add detail, but are not essential to the sentence’s grammar.
  \item Sentences remain complete without them (they still have a subject and verb).
  \item On the SAT, recognizing prepositional phrases helps:
  \begin{itemize}
    \item Avoid agreement errors (ignore prepositional phrases when matching subject–verb).
    \item Strip down to the core sentence quickly for accuracy.
  \end{itemize}
\end{itemize}

\vspace{0.5em}
\textbf{Example (core sentence):}  
The scent was of fatty crabs.
\end{frame}

% ===== Frame 4 =====
\begin{frame}{Quick Exercise: Strip It Down}
\small
\textbf{Originals:}
\begin{enumerate}
  \item \sout{Throughout the living room} was the scent of fatty crabs \sout{that had expired weeks ago}.
  \item I put my sister \sout{on the diet} after it worked so well \sout{for me}.
\end{enumerate}

\vspace{0.8em}
\textbf{Core Sentences:}
\begin{enumerate}
  \item The scent was of fatty crabs.
  \item I put my sister on a diet after it worked.
\end{enumerate}
\end{frame}

\begin{frame}{Prepositional Phrases: Practice Sentences}
\small
\begin{enumerate}
  \item Hillary got into the boat for the short trip to Haiti.
  \item If you do business with me, you’ll never get the better end of the deal.
  \item We’ll need to see the receipts for the underwear you bought on Monday.
  \item I drove by my house to check if the package from Amazon had arrived.
  \item The eleven robbers broke into the casino vault with their perfectly executed plan.
  \item Since the hypothesis of string theory, scientists have been back at the drawing board.
\end{enumerate}
\end{frame}

\begin{frame}{Prepositional Phrases: Answer Key}
\small
\begin{enumerate}
  \item Hillary got \sout{into the boat for the short trip to Haiti}.  
        \textbf{Core:} Hillary got.

  \item If you do business \sout{with me}, you’ll never get the better end \sout{of the deal}.  
        \textbf{Core:} If you do business, you’ll never get the better end.

  \item We’ll need to see the receipts \sout{for the underwear} you bought \sout{on Monday}.  
        \textbf{Core:} We’ll need to see the receipts you bought.

  \item I drove \sout{by my house} to check if the package \sout{from Amazon} had arrived.  
        \textbf{Core:} I drove to check if the package had arrived.

  \item The eleven robbers broke \sout{into the casino vault with their perfectly executed plan}.  
        \textbf{Core:} The eleven robbers broke.

  \item \sout{Since the hypothesis of string theory}, scientists have been back \sout{at the drawing board}.  
        \textbf{Core:} Scientists have been back.
\end{enumerate}
\end{frame}

% ===== Frame 1 =====
\begin{frame}
  \centering
  \Huge \textbf{Subject–Verb Agreement} \\[1.5em]
  \Large Unlocking the Essence
\end{frame}

% ===== Frame 2 =====
\begin{frame}{Definition \& Simple Examples}
\small
\textbf{Definition:} The verb must agree in number with its subject.  
\begin{itemize}
  \item Singular subject → singular verb  
  \item Plural subject → plural verb
\end{itemize}

\vspace{0.5em}
\textbf{Example 1}  
Wrong: You \sout{is} smart.  
Correct: You \textbf{are} smart.

\vspace{0.5em}
\textbf{Example 2}  
Wrong: Everyday the alarm clock goes off and we \sout{wakes} up to confront our lives.  
Correct: Everyday the alarm clock goes off and we \textbf{wake} up to confront our lives.
\end{frame}

% ===== Frame 3 =====
\begin{frame}{Why It’s Important on the SAT}
\small
\textbf{Main Takeaway}
\textbf{}
\begin{itemize}
  \item Subject = the “doer” or main feature (a noun).  
  \item Verb = the action word.  
  \item On the SAT, prepositional phrases often distract from the true subject.  
  \item Strategy: \textbf{Cross out prepositional phrases} to reveal the subject.
\end{itemize}

\vspace{0.8em}
\textbf{Example:}
\begin{itemize}
    \item \textbf{Original sentence:} Investigations into the scandal (shows/show) a lot more than we want to know.
    \item \textbf{Cleaned sentence:}Investigations \sout{into the scandal} (shows/show) a lot more than we want to know.  
    \item \textbf{Core subject:} Investigations (plural) → correct verb is \textbf{show}.
\end{itemize}

\end{frame}

% ===== Frame: Examples 3 & 4 =====
\begin{frame}{Subject–Verb Agreement: Examples 3 \& 4}
\small
\textbf{Example 3} \\
Question: Films by Miyazaki and Itami, including Miyazaki's \textit{Spirited Away}, (excites/excite) the imagination. \\
Step 1: Cross out extras → Films \sout{by Miyazaki and Itami, including Miyazaki's \textit{Spirited Away}} (excites/excite) the imagination. \\
Step 2: Subject = \textbf{Films} \\
Step 3: Plural → verb = \textbf{excite} \\
Answer: Films by Miyazaki and Itami, including Miyazaki's \textit{Spirited Away}, \textbf{excite} the imagination.  

\vspace{0.8em}
\textbf{Example 4} \\
Question: Her jewelry, in addition to her Pokémon cards, (was/were) stolen by the robber. \\
Step 1: Cross out extras → Her jewelry \sout{, in addition to her Pokémon cards,} (was/were) stolen by the robber. \\
Step 2: Subject = \textbf{Her jewelry} \\
Step 3: Singular → verb = \textbf{was} \\
Answer: Her jewelry, in addition to her Pokémon cards, \textbf{was} stolen by the robber.
\end{frame}

% ===== Frame: Examples 5 & 6 =====
\begin{frame}{Subject–Verb Agreement: Examples 5 \& 6}
\small
\textbf{Example 5} \\
Question: Beside the bins, where one could smell the stench of rotten eggs, (was/were) a pack of philosophy majors gathering cans. \\
Step 1: Cross out extras →  Beside the bins, where one could smell the stench of rotten eggs, (was/were) a pack \sout{of philosophy majors gathering cans}. \\
Step 2: Subject = \textbf{a pack} \\
Step 3: Singular → verb = \textbf{was} \\
Answer: Beside the bins, where one could smell the stench of rotten eggs, \textbf{was} a pack of philosophy majors gathering cans.  

\vspace{0.8em}
\textbf{Example 6} \\
Question: Inside heaven’s kingdom (rests/rest) Charlie and his angels. \\
Step 1: Cross out extras → (rests/rest) \textbf{Charlie and his angels}. \\
Step 2: Subject = \textbf{Charlie and his angels} \\
Step 3: Plural → verb = \textbf{rest} \\
Answer: Inside heaven’s kingdom \textbf{rest} Charlie and his angels.
\end{frame}

% ===== Frame: Examples 8 & 9 =====
\begin{frame}{Subject–Verb Agreement: Examples 8 \& 9}
\small
\textbf{Example 7} \\
Question: There (is/are) many other examples to support my point. \\
Step 1: No prepositional phrases to cross out. \\
Step 2: Subject = \textbf{many other examples} \\
Step 3: Plural → verb = \textbf{are} \\
Answer: There \textbf{are} many other examples to support my point.

\textbf{Example 8} \\
Question: The few ideas \sout{that I’ve come up with last night} (has/have) given my team enough to work with. \\
Step 1: Cross out extras → The few ideas \sout{that I’ve come up with last night} (has/have) given my team enough to work with. \\
Step 2: Subject = \textbf{The few ideas} \\
Step 3: Plural → verb = \textbf{have} \\
Answer: The few ideas \sout{that I’ve come up with last night} \textbf{have} given my team enough to work with.


\end{frame}

% ===== Frame: Example 10 =====
\begin{frame}{Subject–Verb Agreement: Example 10}

\small
\textbf{Example 9} \\
Question: The forks and knives are in the kitchen, and the jar \sout{with the Thai peanut sauce} (has/have) been sitting in the refrigerator. \\
Step 1: Cross out extras → the jar \sout{with the Thai peanut sauce} (has/have) been sitting in the refrigerator. \\
Step 2: Subject = \textbf{the jar} \\
Step 3: Singular → verb = \textbf{has} \\
Answer: The forks and knives are in the kitchen, and the jar \sout{with the Thai peanut sauce} \textbf{has} been sitting in the refrigerator.

\vspace{0.9em}
\textbf{Example 10} \\
Question: The players \sout{on our all-star tennis team} (is/are) taken on luxury cruises every year. \\
Step 1: Cross out extras → The players \sout{on our all-star tennis team} (is/are) taken on luxury cruises every year. \\
Step 2: Subject = \textbf{The players} \\
Step 3: Plural → verb = \textbf{are} \\
Answer: The players \sout{on our all-star tennis team} \textbf{are} taken on luxury cruises every year.
\end{frame}

% ===== Frame: Example 11 =====
\begin{frame}{Subject–Verb Agreement: Example 11}
\small
\textbf{Example 11} \\
Question: Where are the cookies that (was/were) in the cookie jar? \\
Answer: Where are the cookies that \textbf{were} in the cookie jar? \\

\vspace{0.5em}
Note: The subject is \textbf{cookies}, which is plural. The relative clause \emph{that were in the cookie jar} describes them.
\end{frame}

% ===== Frame: Example 12 =====
\begin{frame}{Subject–Verb Agreement: Example 12}
\small
\textbf{Example 12} \\
Question: I have no interest in luxury products, which (caters/cater) only to the wealthy. \\
Answer: I have no interest in luxury products, which \textbf{cater} only to the wealthy.  

\vspace{0.8em}
\textbf{Trickier Example (two verbs):} \\
Mastery of magic tricks that truly (surprises/surprise) the audience (requires/require) lots of time.  

\begin{itemize}
  \item Subject for \textbf{requires/require} = Mastery (singular) → \textbf{requires}.  
  \item Subject for \textbf{surprises/surprise} = Magic tricks (plural) → \textbf{surprise}.
\end{itemize}

Answer: Mastery of magic tricks that truly \textbf{surprise} the audience \textbf{requires} lots of time.
\end{frame}

% ===== Frame: Examples 13–15 =====
\begin{frame}{Subject–Verb Agreement: Examples 13–15}
\small
\textbf{Example 13} \\
Question: \textit{The Simpsons (is/are)} the longest running American sitcom. \\
Answer: \textit{The Simpsons \textbf{is}} the longest running American sitcom. \\
Rule: Names of books, TV shows, bands, and movies are singular.

\vspace{0.7em}
\textbf{Example 14} \\
Question: Charles and Kate (was/were) at the ball last night. \\
Answer: Charles and Kate \textbf{were} at the ball last night. \\
Rule: Subjects joined by \textbf{and} are always plural.

\vspace{0.7em}
\textbf{Example 15} \\
Question: Everybody (loves/love) Raymond. \\
Answer: Everybody \textbf{loves} Raymond. \\
Rule: Words like \textbf{everybody, everything, each, anybody, no one} are all singular subjects.
\end{frame}

% ===== Frame: Example 16 =====
\begin{frame}{Subject–Verb Agreement: Example 16}
\small
\textbf{Example 16} \\
Questions:  
Each of the candidates (has/have) two minutes to respond. \\
Neither of the candidates (wants/want) to respond.  

\vspace{0.5em}
\textbf{Answers:}  
Each of the candidates \textbf{has} two minutes to respond. \\
Neither of the candidates \textbf{wants} to respond.  

\vspace{0.5em}
\textbf{Rule:} \emph{Each, neither, and either are all singular subjects.}
\end{frame}

% ===== Frame: Example 17 =====
\begin{frame}{Subject–Verb Agreement: Example 17}
\small
\textbf{Wrong:} John and Harry studied computer science and \sout{was} recruited by Google to develop new services.  

\textbf{Sentence 1:} John and Harry studied computer science. \textbf{Correct.}  

\textbf{Sentence 2:} John and Harry \sout{was} recruited by Google to develop new services. \textbf{Wrong.}  

\textbf{Corrected:} John and Harry studied computer science and \textbf{were} recruited by Google to develop new services.
\end{frame}

% ===== Frame: Example 18 =====
\begin{frame}{Subject–Verb Agreement: Example 18}
\small
\textbf{Wrong:} Poisonous traps that \sout{attracts} and then \sout{kills} off rats are spread throughout this office.  

\textbf{Sentence 1:} Poisonous traps that \sout{attracts} rats are spread throughout this office. \textbf{Wrong.}  

\textbf{Sentence 2:} Poisonous traps that then \sout{kills} off rats are spread throughout this office. \textbf{Wrong.}  

\textbf{Corrected:} Poisonous traps that \textbf{attract} and then \textbf{kill} off rats are spread throughout this office.
\end{frame}

% ===== Frame: Example 19 =====
\begin{frame}{Subject–Verb Agreement: Example 19}
\small
\textbf{Wrong:} I was walking down the street and \sout{were} chatting with my friend about his day.  

\textbf{Sentence 1:} I was walking down the street. \textbf{Correct.}  

\textbf{Sentence 2:} I \sout{were} chatting with my friend about his day. \textbf{Wrong.}  

\textbf{Corrected:} I was walking down the street and (was) chatting with my friend about his day.  

\vspace{0.5em}
\textbf{Note:} The second \emph{was} is unnecessary — the first \emph{was} already acts as the helping verb for both \emph{walking} and \emph{chatting}.  
Stripped down, the sentence reads: \emph{I was walking and chatting}, which is correct.
\end{frame}

% ===== Frame 1: Present Simple =====
\begin{frame}{Tenses: Present Simple}
\small
\textbf{Form:} Subject + base verb (+ s for he/she/it).  

\textbf{Examples:}
\begin{itemize}
  \item He hugs.
  \item He swims.
  \item He is.
\end{itemize}

\textbf{Use:}
\begin{itemize}
  \item General truths (\emph{The sun rises in the east.})
  \item Habits (\emph{She walks to school every day.})
  \item States (\emph{He is happy.})
\end{itemize}
\end{frame}

% ===== Frame 2: Past Simple =====
\begin{frame}{Tenses: Past Simple}
\small
\textbf{Form:} Subject + past form of the verb.  

\textbf{Examples:}
\begin{itemize}
  \item He hugged.
  \item He swam.
  \item He was.
\end{itemize}

\textbf{Use:}
\begin{itemize}
  \item Completed actions in the past (\emph{She finished her homework yesterday.})
  \item Specific time markers: yesterday, last year, in 2010.
\end{itemize}
\end{frame}

% ===== Frame 3: Future Simple =====
\begin{frame}{Tenses: Future Simple}
\small
\textbf{Form:} Subject + will + base verb.  

\textbf{Examples:}
\begin{itemize}
  \item He will hug.
  \item He will swim.
  \item He will be.
\end{itemize}

\textbf{Use:}
\begin{itemize}
  \item Predictions (\emph{It will rain tomorrow.})
  \item Promises (\emph{I will help you.})
  \item Spontaneous decisions (\emph{I’ll call her now.})
\end{itemize}
\end{frame}

% ===== Frame 4: Present Perfect =====
\begin{frame}{Tenses: Present Perfect}
\small
\textbf{Form:} Subject + has/have + past participle.  

\textbf{Examples:}
\begin{itemize}
  \item He has hugged.
  \item He has swum.
  \item He has been.
\end{itemize}

\textbf{Use:}
\begin{itemize}
  \item Actions with present relevance (\emph{I have lost my keys.})
  \item Life experiences (\emph{She has visited Japan.})
  \item Actions that started in the past and continue now (\emph{We have lived here for 5 years.})
\end{itemize}

\textbf{Note:} On the SAT, answer choices with Present Perfect are almost always wrong!
\end{frame}

% ===== Frame 5: Past Perfect =====
\begin{frame}{Tenses: Past Perfect}
\small
\textbf{Form:} Subject + had + past participle.  

\textbf{Examples:}
\begin{itemize}
  \item He had hugged.
  \item He had swum.
  \item He had been.
\end{itemize}

\textbf{Use:}
\begin{itemize}
  \item To show one action happened before another past action (\emph{She had finished dinner before he arrived.})
  \item Sequencing past events clearly.
\end{itemize}

\textbf{Note:} On the SAT, Past Perfect is rarely correct — choices with it are almost always wrong.
\end{frame}

% ===== Frame: Examples 1–2 =====
\begin{frame}{Tense Consistency: Examples 1–2}
\small
\textbf{Example 1} \\
Wrong: Whenever we \sout{stopped} by the market, my mom always tries to negotiate the prices. \\
Correct: Whenever we stop by the market, my mom always \textbf{tries} to negotiate the prices. \\
Correct: Whenever we stopped by the market, my mom always \textbf{tried} to negotiate the prices.  

\vspace{0.8em}
\textbf{Example 2} \\
Wrong: After winning Wimbledon in 2012, Federer \sout{regained} the top ranking and \sout{declares} himself the best in the world. \\
Correct: After winning Wimbledon in 2012, Federer \textbf{regained} the top ranking and \textbf{declared} himself the best in the world.
\end{frame}

% ===== Frame: Examples 3–4 =====
\begin{frame}{Tense Consistency: Examples 3–4}
\small
\textbf{Example 3} \\
Wrong: The end of World War II came when German forces \sout{surrender} in Berlin and Italy. \\
Correct: The end of World War II came when German forces \textbf{surrendered} in Berlin and Italy. \\
Rule: Historical events usually require past tense.  

\vspace{0.8em}
\textbf{Example 4} \\
Wrong: Although the cheetah \sout{holds} the record for fastest land animal, many other mammals \sout{outlasted} it. \\
Correct: Although the cheetah \textbf{holds} the record for fastest land animal, many other mammals \textbf{outlast} it. \\
Rule: Facts or universal truths must be in the present tense.
\end{frame}

% ===== Frame: Example 5 =====
\begin{frame}{Tense Consistency: Example 5}
\small
\textbf{Example 5} \\
Correct: When I was young, I hated vegetables, but now I \textbf{love} them. \\
Correct: Because he was late for the anniversary dinner, she is thinking about leaving him.  

\vspace{0.5em}
Note: Sometimes two different tenses are correct if the meaning requires two time periods.
\end{frame}

% ===== Frame 1: Title =====
\begin{frame}
  \centering
  \Huge \textbf{Point of View (POV)}\\[1em]
  \Large Keep POV consistent within sentences and paragraphs
\end{frame}

% ===== Frame 2: Rule + Example 1 =====
\begin{frame}{POV Rule \& Example 1}
\small
\textbf{Rule:} Pronouns must stay in the same person. Don’t shift between \emph{one / you / he or she / they} unless meaning requires it.

\vspace{0.6em}
\textbf{Example 1}
\begin{itemize}
  \item \textbf{Wrong:} If \underline{one} does not believe, \underline{you} will not succeed.
  \item \textbf{Correct:} If \underline{one} does not believe, \underline{one} will not succeed.
  \item \textbf{Also correct:} If \underline{you} do not believe, \underline{you} will not succeed.
\end{itemize}
\end{frame}

% ===== Frame 3: Example 2 =====
\begin{frame}{POV Example 2}
\small
\textbf{Example 2}
\begin{itemize}
  \item \textbf{Wrong:} If \underline{someone} wants to play tennis, \underline{you} should know how to serve.
  \item \textbf{Correct:} If \underline{someone} wants to play tennis, \underline{he or she} should know how to serve.
  \item \textbf{Also correct:} If \underline{you} want to play tennis, \underline{you} should know how to serve.
\end{itemize}
\end{frame}

% ===== Frame 4: Quick Exercise (no answers) =====
\begin{frame}{POV Consistency — Quick Exercise}
\small
Keep the point of view the same in each sentence. (Answers may vary.)
\begin{enumerate}
  \item The flight attendants demanded that \underline{we} leave the plane even though \underline{you} wanted to finish the movie.
  \item Despite how hard salesmen try, sometimes \underline{you} just can’t get anyone \underline{you} want to buy a house.
\end{enumerate}
\textit{Rewrite each so the POV is consistent (e.g., all “we,” all “you,” or all third person).}
\end{frame}

% ===== Frame 1: Title =====
\begin{frame}
  \centering
  \Huge \textbf{Combining Sentences} \\[1em]
  \Large Techniques for Concise, Clear Writing
\end{frame}

% ===== Frame 2: Technique Intro =====
\begin{frame}{Technique 1: Use a Trailing Phrase}
\small
\textbf{Definition:} Attach the second sentence as a trailing modifier/phrase to the first.  
This is the most commonly tested method on the SAT.  

\textbf{Why it works:}
\begin{itemize}
  \item Removes repetition
  \item Keeps focus on the main idea
  \item Improves sentence flow
\end{itemize}
\end{frame}

% ===== Frame 3: Example 1 =====
\begin{frame}{Example 1: Trailing Phrase}
\small
\textbf{Before:}  
To get the gun-control law passed, the President pointed out the numerous shootings that happen every year.  
He illustrated the dangers of having few restrictions.  

\vspace{0.5em}
\textbf{After (combined):}  
To get the gun-control law passed, the President pointed out the numerous shootings that happen every year, \textbf{illustrating the dangers of having few restrictions}.
\end{frame}

% ===== Frame 4: Example 2 =====
\begin{frame}{Example 2: Trailing Phrase}
\small
\textbf{Before:}  
Students often see the complex theorems of physics as useless and tiresome.  
They don’t know that every piece of modern technology is founded on the discoveries of quantum physics.  

\vspace{0.5em}
\textbf{After (combined):}  
Students often see the complex theorems of physics as useless and tiresome, \textbf{not knowing that every piece of modern technology is founded on the discoveries of quantum physics}.
\end{frame}

% ===== Frame: Example 3 =====
\begin{frame}{Example 3: Trailing Phrase (continued)}
\small
\textbf{Before:}  
The monkey was tied down because other animals were distracted by its eating habits.  
It was not because of its tendency to escape.  

\vspace{0.5em}
\textbf{After (combined):}  
The monkey was tied down because other animals were distracted by its eating habits, \textbf{not because of its tendency to escape}.
\end{frame}

% ===== Frame: Example 4 =====
\begin{frame}{Example 4: Trailing Phrase (continued)}
\small
\textbf{Before:}  
Inside the dusty cabinet was a map of the Underground Railroad.  
It was a network of underground tunnels slaves once used to escape from the South.  

\vspace{0.5em}
\textbf{After (combined):}  
Inside the dusty cabinet was a map of the Underground Railroad, \textbf{a network of underground tunnels slaves once used to escape from the South}.
\end{frame}

% ===== Frame: Method 2 =====
\begin{frame}{Technique 2: Use a Preposition}
\small
\textbf{Definition:} Link the second idea to the first using a preposition (\emph{with, by, through, as, etc.}).  

\textbf{Why it works:}
\begin{itemize}
  \item Reduces wordiness  
  \item Shows relationship between the two ideas  
  \item Keeps sentence compact
\end{itemize}
\end{frame}

% ===== Frame: Example 5 =====
\begin{frame}{Example 5: Use a Preposition}
\small
\textbf{Before:}  
Joseph finished his homework. His teacher helped him.  

\vspace{0.5em}
\textbf{After (combined):}  
Joseph finished his homework \textbf{with the help of his teacher}.
\end{frame}

% ===== Frame: Example 6 =====
\begin{frame}{Example 6: Use a Preposition}
\small
\textbf{Before:}  
He is one of the fastest runners in the world.  
His accomplishments are demonstrated by his numerous world records.  

\vspace{0.5em}
\textbf{After (combined):}  
He is one of the fastest runners in the world \textbf{as demonstrated by his numerous world records}.
\end{frame}

% ===== Technique 3: Turn one into a dependent clause/modifier =====
\begin{frame}{Technique 3: Make a Dependent Clause or Modifier}
\small
\textbf{Idea:} Convert one sentence into a reason/time/condition clause (or a front modifier) to attach to the other.
\end{frame}

% ----- Example 7 -----
\begin{frame}{Example 7 — Dependent Clause}
\small
\textbf{Before:}\\
Jacob has decided to avoid snacks and soda. The reason for the diet is that he wants to lose weight.

\vspace{0.5em}
\textbf{After (combined):}\\
\textbf{Because he wants to lose weight}, Jacob has decided to avoid snacks and soda.
\end{frame}

% ----- Example 8 -----
\begin{frame}{Example 8 — Front Modifier (Appositive-like)}
\small
\textbf{Before:}\\
The giant panda is the rarest bear in the world today. It has large, distinctive, black patches around its eyes, strong jaw muscles, and a long tail.

\vspace{0.5em}
\textbf{After (combined):}\\
\textbf{The rarest bear in the world today}, the giant panda has large, distinctive, black patches around its eyes, strong jaw muscles, and a long tail.
\end{frame}

% ===== Technique 4: Use a conjunction =====
\begin{frame}{Technique 4: Use a Conjunction}
\small
\textbf{Idea:} Join closely related independent clauses with a logical connector (e.g., \emph{but, and, so, yet}).
\end{frame}

% ----- Example 9 -----
\begin{frame}{Example 9 — Conjunction}
\small
\textbf{Before:}\\
On the surface, \textit{Seinfeld} is most famous for its light-hearted dialogue. Included among the many episodes is an assortment of comments on racism, homosexuality, and death.

\vspace{0.5em}
\textbf{After (combined):}\\
On the surface, \textit{Seinfeld} is most famous for its light-hearted dialogue, \textbf{but} included among the many episodes is an assortment of comments on racism, homosexuality, and death.
\end{frame}

% ===== Technique 5: Link two verbs with \emph{and} =====
\begin{frame}{Technique 5: Link Two Verbs with \emph{and}}
\small
\textbf{Idea:} Keep the same subject; connect actions with \emph{and} to avoid repetition.
\end{frame}

% ----- Example 10 -----
\begin{frame}{Example 10 — Link Verbs}
\small
\textbf{Before:}\\
The people sitting in front of me on the train were talking throughout the ride. They would not turn their cell phones off even after being told to do so.

\vspace{0.5em}
\textbf{After (combined):}\\
The people sitting in front of me on the train \textbf{were talking throughout the ride and would not turn} their cell phones off even after being told to do so.
\end{frame}

% ===== Technique 6: Use a relative clause =====
\begin{frame}{Technique 6: Use a Relative Clause}
\small
\textbf{Idea:} Turn the second sentence into a clause beginning with \emph{who/whom/whose/which/that} to describe a noun in the first.
\end{frame}

% ----- Example 11 -----
\begin{frame}{Example 11 — Relative Clause}
\small
\textbf{Before:}\\
John Durgin worked as an accountant for ten years and then became a math teacher. He first learned to calculate in his head by reciting multiplication tables at home.

\vspace{0.5em}
\textbf{After (combined):}\\
\textbf{John Durgin, who worked as an accountant for ten years and then became a math teacher,} first learned to calculate in his head by reciting multiplication tables at home.
\end{frame}

% ----- Example 12 -----
\begin{frame}{Example 12 — Relative Clause}
\small
\textbf{Before:}\\
Every car is powered by an engine. The engine converts fuel and heat into mechanical motion.

\vspace{0.5em}
\textbf{After (combined):}\\
Every car is powered by an engine, \textbf{which converts} fuel and heat into mechanical motion.
\end{frame}

% ===== Technique 7: Use an infinitive to express purpose =====
\begin{frame}{Technique 7: Infinitive of Purpose}
\small
\textbf{Idea:} Replace o that / in order that” style second sentences with \emph{to + verb}.
\end{frame}

% ----- Example 13 -----
\begin{frame}{Example 13 — Infinitive of Purpose}
\small
\textbf{Before:}\\
The little boy happily ran home. He would tell his mom he had found the last golden ticket.

\vspace{0.5em}
\textbf{After (combined):}\\
The little boy happily ran home \textbf{to tell} his mom he had found the last golden ticket.
\end{frame}

% ===== Frame 1: Guideline 1 =====
\begin{frame}{Guideline 1: Fewer Pronouns is er}
\small
\textbf{Tip:} Avoid unnecessary pronouns (especially \emph{this, these, they, it}).  

\vspace{0.5em}
\textbf{Example 14}  
The arctic owl’s coat of snow-white feathers acts as camouflage. It keeps the owl hidden by blending the animal in with its surroundings.  

\textbf{Answer:}  
D) camouflaging, keeping — best because it avoids the unnecessary pronoun \emph{it}.
\end{frame}

% ===== Frame 2: Guideline 2 =====
\begin{frame}{Guideline 2: Keep the Intended Meaning}
\small
\textbf{Tip:} Donâge the meaning of the original sentence when combining.  

\vspace{0.5em}
\textbf{Example 15}  
Chinese families get ready for Mid-Autumn Festival by doing several things. They light lanterns, prepare mooncakes, and arrange flowers.  

\textbf{Answer:}  
C) Chinese families get ready for Mid-Autumn Festival by lighting lanterns, preparing mooncakes, and arranging flowers.  

\vspace{0.5em}
\textbf{Why:} Options B and D alter the meaning — they describe “how” families get ready, not the intended “whon C preserves intent.
\end{frame}

% ===== Frame 3: Guideline 3 =====
\begin{frame}{Guideline 3: Avoid Repeated Words}
\small
\textbf{Tip:} Eliminate redundancy to keep sentences concise.  

\vspace{0.5em}
\textbf{Example 16}  
The restaurant was highly recommended by the food critics. These critics stayed there for four hours to savor every dish.  

\textbf{Answer:}  
B) critics, who — avoids repeating “critics.”  

\vspace{0.5em}
\textbf{Rule:} Whenever you see repeated words in combining answer ch the best option is usually the one that removes the repetition.
\end{frame}

% ===== Frame: Guideline 4 =====
\begin{frame}{Guideline 4: The Less Words, The Better}
\small
\textbf{Tip:} More words usually mean more complexity. On the SAT, the best combined sentence is typically the most concise version that still keeps the meaning.

\vspace{0.5em}
\textbf{Example 17}  
The new hardware runs at a faster rate when compared to the old one. This increased speed reduces costs in our technology department, where we have the most overhead.  

\textbf{Answer Choices:}  
A) NO CHANGE \\
B) When compared to the old one, the new hardware runs at a faster rate, which reduces costs in our technology department, where we have the most overhead. \\
C) The new hardware runs at a faster rate when compared to the old one, and this increased speed reduces costs in our technology department, where we have the most overhead. \\
D) The new hardware runs at a faster rate when compared to the old one; by increasing the speed, we reduce costs in our technology department, where we have the most overhead.  

\vspace{0.5em}
\textbf{Correct Answer: B} — eliminates unnecessary wording (\emph{increased speed}) and is more concise.
\end{frame}

% ===== Frame: Checklist for Combining Sentences =====
\begin{frame}{Checklist: Combining Sentences}
\small
When choosing the best way to combine sentences on the SAT, check for these:

\begin{enumerate}
  \item \textbf{Fewer Pronouns}  
  Avoid unnecessary words like \emph{this, these, they, it} uess they are clear.

  \item \textbf{Keep the Intended Meaning}  
  Don’t change the original meaning when combining. Preserve the author’s intent.

  \item \textbf{Avoid Repeated Words}  
  Cut out redundancy. If two choices say the same word multiple times, the shorter is better.

  \item \textbf{The Less Words, The Better}  
  The most concise correct version is usually right — eliminate clutter while keeping clarity.
\end{enumerate}

\vspace{0.6em}
\textbf{Golden Rule:} The best answer will be conclear, and faithful to the original meaning.
\end{frame}



% ===== Frame: Final Summary =====
\begin{frame}{Summary \& Final Thoughts}
\small
\textbf{Key Takeaways from SAT Writing:}
\begin{itemize}
  \item \textbf{Concision, Clarity, Accuracy} — shorter, precise, and grammatically correct answers win.
  \item \textbf{Grammar Rules} — subject–verb agreement, verb tense consistency, pronoun reference, modifiers.
  \item \textbf{Sentence Combining} — use trailing phrases, prepositions, dependent claujunctions, relative clauses, and infinitives effectively.
  \item \textbf{Guidelines} — fewer pronouns, preserve meaning, avoid repetition, and choose the most concise version.
\end{itemize}

\vspace{0.6em}
\textbf{Final Advice:}  
Strip away distractions, focus on the core sentence, and trust the simplest, clearest choice.

\vspace{1em}
\centering
\Large \textbf{Congrats on getting through your first slides!!} \\
\large You’ve got this!
\end{frame}


\end{document}

